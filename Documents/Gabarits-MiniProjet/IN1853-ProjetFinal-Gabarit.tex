\documentclass[a4paper,12pt]{article}
% Packages nécessaires  
\usepackage[utf8]{inputenc}
\usepackage[margin=1.75cm]{geometry}
\usepackage[T1]{fontenc}
\usepackage[french]{babel}
\usepackage{color}
\usepackage{hyperref}
\usepackage{graphicx}   
\usepackage{booktabs}    
\usepackage{hyperref}    
\usepackage{amsmath}     
\usepackage{amsfonts}    
\usepackage{amssymb}     
\usepackage{mathrsfs}    
\usepackage{tikz}        
\usepackage{multirow}    
\usepackage{multicol}    


\begin{document}

% Titre du document
\title{\textbf{Titre du Projet}}

\author{
    \small 
    \begin{tabular}{ccc}
        \textbf{Auteur 1} & \textbf{Auteur 2} & \textbf{Auteur 3} \\
        ABCD75350201 & EFGH75350202 & IJKL75350203 \\
        \textit{Email1@uqo.ca} & \textit{Email2@uqo.ca} & \textit{Email3@uqo.ca} \\
    \end{tabular}
}
		
\normalsize
\date{}
\maketitle

% Abstract
\begin{abstract}
    Voici un résumé succinct de votre rapport. Cet abstract devrait fournir un aperçu des objectifs, méthodes, résultats et conclusions de votre étude. Le résume doit avoir au plus 150 mots.
\end{abstract}

% Mots clés
\textbf{Mots clés :} Intelligence artificielle, évaluation comparative, algorithmes, rapport étudiant.

% Début du texte principal
\section{Introduction}
Ce rapport présente une évaluation comparative des algorithmes d'intelligence artificielle \cite{nom2023}. L'objectif est de comprendre les performances de divers algorithmes dans des contextes spécifiques \cite{article1,book1}.

\section{Évaluation des algorithmes}
\begin{table}[ht]
    \centering
    \caption{Évaluation comparative des algorithmes IA}
    \begin{tabular}{@{}lll@{}}
        \toprule
        \textbf{Algorithme} & \textbf{Précision} & \textbf{Temps d'exécution} \\ \midrule
        Algorithme A & 85\% & 10s \\
        Algorithme B & 90\% & 8s \\
        Algorithme C & 75\% & 12s \\
        Algorithme D & 92\% & 7s \\ \bottomrule
    \end{tabular}
\end{table}

\begin{figure}[ht]
    \centering
    \includegraphics[width=0.7\textwidth]{img/im1.png}
    \caption{SVM avec noyau linéaire}
\end{figure}

% Exemples d'équations
\section{Exemples d'équations mathématiques}
Voici quelques exemples d'équations mathématiques :

\begin{align}
    E = mc^2 \tag{1} \\
    a^2 + b^2 = c^2 \tag{2}
\end{align}

% Bibliographie
\newpage
\bibliographystyle{unsrt}
\bibliography{references}

\end{document}